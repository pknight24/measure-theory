\documentclass{article}

\usepackage{amssymb,amsmath,amsthm}


%%% some helpful shortcuts
\newcommand{\bbR}{\mathbb{R}}

\newtheorem{theorem}{Theorem}[section]
\newtheorem{corollary}[theorem]{Corollary}
\newtheorem{proposition}[theorem]{Proposition}
\newtheorem{definition}[theorem]{Definition}
\newtheorem{example}[theorem]{Example}


\title{Measure Theory Notes}
\author{Parker Knight}

\begin{document}

\maketitle
\tableofcontents
\newpage


\section{Introduction}

This document contains a collection of notes on measures and integrals,
mostly inspired by the text \textit{Elements of Integration} by Bartle. This
is primarily a learning tool for myself; I'm currently learning this material in
preparation for my PhD program which begins this fall.

In this section I provide some background definitions which may be referenced throughout
the notes.

\subsection{Background}

\newpage

\section{$\sigma$-Algebras and Measurable Functions} 

Loosely speaking, a measure is a nonnegative, countably additive real-valued
function defined on a collection of
well-behaved sets. Before we can give a more rigorous definition of a measure,
we need to examine what is precisely meant by \textit{well-behaved}. This
section explores the concept of the $\sigma$-algebra, a collection of sets which
holds certain properties required for the formal definition of measure.
I will also discuss measurable functions, and give examples of both.

\subsection{$\sigma$-Algebras}

\begin{definition}[$\sigma$-algebra]\label{def:sigma-alg}
    Let $X$ be any set, and let $\Sigma$ be a collection of subsets of $X$.
    We say that $\Sigma$ is a \textit{$\sigma$-algebra} if:
    \begin{enumerate}
        \item $\emptyset, X \in \Sigma$
        \item If $A \in \Sigma$, then $A^c \in \Sigma$
        \item If $(A_n)$ is a sequence of sets in $\Sigma$, then $\bigcup A_n
        \in \Sigma$
    \end{enumerate}
    
\end{definition}

Defintion \ref{def:sigma-alg} states that a $\sigma$-algebra is closed under
taking complements and unions. It follows by De Morgan's laws that every
$\sigma$-algebra is also closed under taking intersections. Below provides some
simple examples.

\begin{example}
    If $X$ is any set, then $\Sigma = \{ X, \emptyset \}$ is the trivial
    $\sigma$-algebra.
\end{example}

\begin{example}
    For any $X$, the power set of $X$ (the set of all subsets of $X$) is a
    $\sigma$-algebra.
\end{example}

\begin{example}
    For any $X$ and $A \subset X$, the set $\Sigma = \{\emptyset, A, A^c, X \}$
    is a $\sigma$-algebra.
\end{example}


A useful result is that the intersection of two $\sigma$-algebras is itself a
$\sigma$-algebra.

\begin{proposition}\label{prop:intersection-sig-alg}
    Let $X$ be an arbitrary set, and let $\Sigma_1$ and $\Sigma_2$ be
    $\sigma$-algebras of $X$. Then $\Sigma_3 = \Sigma_1 \cap \Sigma_2$ is a $\sigma$-algeba.
\end{proposition}

\begin{proof}
    The proof is simple. First, we know that $\emptyset, X \in \Sigma_1$ and
    $\emptyset, X \in \Sigma_2$, so we have $\emptyset, X \in \Sigma_1 \cap
    \Sigma_2$. Now let $A \in \Sigma_1 \cap \Sigma_2$ be arbitrary. Then $A \in
    \Sigma_1$ and $A \in \Sigma_2$, and so we have $A^c \in \Sigma_1$ and $A^c
    \in \Sigma_2$ by Definition \ref{def:sigma-alg}. So $A^c \in \Sigma_1 \cap
    \Sigma_2$. Finally, if $(A_n)$ is a sequence in $\Sigma_1 \cap \Sigma_2$,
    then $(A_n)$ is a sequence in $\Sigma_1$ and $\Sigma_2$ as well. So $\cup An
    \in \Sigma_1$ and $\cup A_n \in \Sigma_2$, so $\cup A_n \in \Sigma_1 \cap \Sigma_2$.
\end{proof}

\section{Measures}

\section{The Lebesgue Integral}

\end{document}