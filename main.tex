\documentclass{amsart}

\usepackage{amssymb,amsmath,amsthm}


\setlength{\parindent}{0cm}

%%% some helpful shortcuts
\newcommand{\bbR}{\mathbb{R}}
\newcommand{\Borel}{\mathcal{B}}

\newtheorem{theorem}{Theorem}[section]
\newtheorem{corollary}[theorem]{Corollary}
\newtheorem{lemma}[theorem]{Lemma}
\newtheorem{proposition}[theorem]{Proposition}
\newtheorem{definition}[theorem]{Definition}
\newtheorem{example}[theorem]{Example}


\title{Measure Theory Notes}
\author{Parker Knight}

\begin{document}

\maketitle
\tableofcontents
\newpage

\section{$\sigma$-Algebras and Measurable Functions} 

Loosely speaking, a measure is a nonnegative, countably additive real-valued
function defined on a collection of
well-behaved sets. Before we can give a more rigorous definition of a measure,
we need to examine what is precisely meant by \textit{well-behaved}. This
section explores the concept of the $\sigma$-algebra, a collection of sets which
holds certain properties required for the formal definition of measure.
I will also discuss measurable functions, and give examples of both.

\subsection{$\sigma$-Algebras}

\begin{definition}[$\sigma$-algebra]\label{def:sigma-alg}
    Let $X$ be any set, and let $\Sigma$ be a collection of subsets of $X$.
    We say that $\Sigma$ is a \textit{$\sigma$-algebra} if:
    \begin{enumerate}
        \item $\emptyset, X \in \Sigma$
        \item If $A \in \Sigma$, then $A^c \in \Sigma$
        \item If $(A_n)$ is a sequence of sets in $\Sigma$, then $\bigcup A_n
        \in \Sigma$
    \end{enumerate}
    
\end{definition}

Defintion \ref{def:sigma-alg} states that a $\sigma$-algebra is closed under
taking complements and unions. It follows by De Morgan's laws that every
$\sigma$-algebra is also closed under taking intersections. For a set $X$ and
$\sigma$-algebra $\Sigma$, the pair $(X, \Sigma)$ is called a \textit{measurable
space}. Below provides some simple examples.

\begin{example}
    If $X$ is any set, then $\Sigma = \{ X, \emptyset \}$ is the trivial
    $\sigma$-algebra.
\end{example}

\begin{example}
    For any $X$, the power set of $X$ (the set of all subsets of $X$) is a
    $\sigma$-algebra.
\end{example}

\begin{example}
    For any $X$ and $A \subset X$, the set $\Sigma = \{\emptyset, A, A^c, X \}$
    is a $\sigma$-algebra.
\end{example}


A useful result is that the intersection of two $\sigma$-algebras is itself a
$\sigma$-algebra.

\begin{proposition}\label{prop:intersection-sig-alg}
    Let $X$ be an arbitrary set, and let $\Sigma_1$ and $\Sigma_2$ be
    $\sigma$-algebras of $X$. Then $\Sigma_3 = \Sigma_1 \cap \Sigma_2$ is a $\sigma$-algeba.
\end{proposition}

\begin{proof}
    The proof is simple. First, we know that $\emptyset, X \in \Sigma_1$ and
    $\emptyset, X \in \Sigma_2$, so we have $\emptyset, X \in \Sigma_1 \cap
    \Sigma_2$. Now let $A \in \Sigma_1 \cap \Sigma_2$ be arbitrary. Then $A \in
    \Sigma_1$ and $A \in \Sigma_2$, and so we have $A^c \in \Sigma_1$ and $A^c
    \in \Sigma_2$ by Definition \ref{def:sigma-alg}. So $A^c \in \Sigma_1 \cap
    \Sigma_2$. Finally, if $(A_n)$ is a sequence in $\Sigma_1 \cap \Sigma_2$,
    then $(A_n)$ is a sequence in $\Sigma_1$ and $\Sigma_2$ as well. So $\cup A_n
    \in \Sigma_1$ and $\cup A_n \in \Sigma_2$, so $\cup A_n \in \Sigma_1 \cap \Sigma_2$.
\end{proof}

Proposition \ref{prop:intersection-sig-alg} yields the following definition.

\begin{definition}
   For any set $X$ and a collection of subsets $\mathcal{A}$ of $X$, the
   $\sigma$-algebra generated by $\mathcal{A}$ is the intersection of all
   $\sigma$-algebras that contain $\mathcal{A}$. By Proposition
   \ref{prop:intersection-sig-alg}, this intersection is itself a $\sigma$-algebra.
\end{definition}

We can also think of the $\sigma$-algebra generated by a collection of subsets
$\mathcal{A}$ as the \textit{smallest} $\sigma$-algebra containing
$\mathcal{A}$.

\begin{proposition}\label{prop:generated-sig-alg-operations}
    If $\mathcal{A}$ is a collection of subsets of $X$ and $\Sigma$ is the
    $\sigma$-algebra generated by $\mathcal{A}$, then every element of $\Sigma$
    can be constructed as a countable union, intersection, or complement of
    elements of $\mathcal{A}$.
\end{proposition}

The proof of this proposition is clear: the set of countable unions,
intersections, and complements of $\mathcal{A}$ is $\sigma$-algebra that
contains $\mathcal{A}$ as long as
it also contains $X$ and $\emptyset$, and so the $\sigma$-algebra generated by
$\mathcal{A}$ must be a subset of this set by definition.

\begin{definition}[Borel algebra]\label{def:borel}
    Let $X = \bbR$. The Borel $\sigma$-algebra $\Borel$ is the $\sigma$-algebra
    generated by all open intervals of the form $(a,b) \subset \bbR$. The
    elements of $\Borel$ are called Borel sets.
\end{definition}

We can show that $\Borel$ is also generated by all closed intervals, as well as
half-open intervals $(a, b]$ and half-rays $(a, \infty)$.

\begin{proposition}\label{prop:borel-generated}
   The Borel algebra $\Borel$ is also generated by
   \begin{enumerate}
       \item The closed intervals $[a,b]$ \label{prop:borel-generated-1}
       \item The half-open intervals $(a,b]$ \label{prop:borel-generated-2}
       \item The half-rays $(a, \infty)$ \label{prop:borel-generated-3}
   \end{enumerate} 
\end{proposition}
\begin{proof}
    For the proof of (\ref{prop:borel-generated-1}), let $\Borel'$ denote the
    $\sigma$-algebra generated by closed intervals. Notice that we can write any
    closed interval $[a,b]$ as $[a,b] =
    \cap_{n=1}^\infty(a - \frac{1}{n}, b + \frac{1}{n})$. So $[a,b]$ is an
    intersection of open intervals, and thus $[a,b] \in \Borel$. Since every
    element of $\Borel'$ can be constructed by countable union, intersection, or
    complement of closed intervals by Proposition
    \ref{prop:generated-sig-alg-operations} and $\Borel$ is closed under these
    operations, it follows that every element of $\Borel'$ is in $\Borel$, and
    thus $\Borel' \subseteq \Borel$. Similarly, we show that $\Borel \subseteq
    \Borel'$ by noting that $(a,b) = \cup_{n=1}^\infty [a+\frac{1}{n}, b -
    \frac{1}{n}]$ and applying the same logic. Thus $\Borel' = \Borel$,
    and so the Borel algebra is generated by closed intervals. 

    \bigskip

    The proofs of (\ref{prop:borel-generated-2}) and
    (\ref{prop:borel-generated-3}) follow similarly. 
\end{proof}

\bigskip

\subsection{Measurable functions}


\begin{definition}[Measurable function]\label{def:measurable-fn}
    Let $(X, \Sigma_1)$ and $(Y, \Sigma_2)$ be measurable spaces, and let $f:X
    \rightarrow Y$ be a function. We say that $f$ is a measurable function if 

    $$ f^{-1}(E) = \{x \in X : f(x) \in E \}  $$

    is in $\Sigma_1$ for every $E \in \Sigma_2$.
\end{definition}

In other words, measurable functions \textit{pull back} measurable sets in
$\Sigma_2$ to sets in $\Sigma_1$. Note the similarity to the definition of
continuous functions between metric spaces. We can give a simpler characterization of
measurable functions when $(Y, \Sigma_2) = (\bbR, \Borel)$, but we first need a
pair of preliminary results.

\begin{lemma}
    Let $(X, \Sigma)$ be a measurable space and let $f : X \rightarrow Y$. Then
    $T = \{E \subseteq Y :  f^{-1}(E) \in \Sigma \}$ is a $\sigma$-algebra.
\end{lemma}
\begin{proof}
Clearly $\emptyset, Y \in T$ since $f^{-1}(\emptyset) = \emptyset \in \Sigma$
and $f^{-1}(Y)  = X \in \Sigma$. Now let $E \in T$ be arbitrary. Then $f^{-1}(E)
\in \Sigma$, so $f^{-1}(E)^c = \{x \in X : f(x) \notin E \} = f^{-1}(E^c) \in
\Sigma$. So $E^c \in T$. Finally  let $(E_n)$ be a sequence of sets in $T$. Then
$(f^{-1}(E_n))$ is a sequence in $\Sigma$, and so $\cup f^{-1}(E_n) \in \Sigma$.
Notice that
$$\bigcup f^{-1}(E_n) = \{x \in X : f(x) \in E_n \textrm{   for some n} \} =
f^{-1}(\cup E_n)$$

So $f^{-1}(\cup E_n) \in \Sigma$, so $\cup E_n \in T$. So $T$ is a $\sigma$-algebra.

\end{proof}

\begin{lemma}\label{lemma:fn-preimage-generated}
   Let $(X, \Sigma)$ be a measurable space and let $f : X \rightarrow Y$. Let
   $\mathcal{A}$ be collection of subsets of $Y$ such that $f^{-1}(E) \in
   \Sigma$ for every $E \in \mathcal{A}$. Then $f^{-1}(F) \in \Sigma$ for every
   $F$ in the $\sigma$-algebra generated by $\mathcal{A}$.
\end{lemma}
\begin{proof}
    Let $\Sigma_{\mathcal{A}}$ denote the $\sigma$-algebra generated by
    $\mathcal{A}$ and let $T = \{E \subseteq Y :  f^{-1}(E) \in \Sigma \}$. Then
    $\mathcal{A} \subset T$ by hypothesis. Since we know by the previous lemma
    that $T$ is a $\sigma$-algebra, then $\Sigma_{\mathcal{A}} \subset T$. So
    for every $F \in \Sigma_{\mathcal{A}}$, we know $F \in T$, and so
    $f^{-1}(F) \in \Sigma$. 
\end{proof}

Now we can state the characterization of measurable functions from $X$ to $\bbR$.

\begin{proposition}\label{prop:measurable-fn-r}
    Let $(X, \Sigma)$ be a measurable space. A function $f:X \rightarrow \bbR$
    is measurable if for every $\alpha > 0$, the set

    $$\{x \in X : f(x) > \alpha \}$$

    is in $\Sigma$.
\end{proposition}
\begin{proof}
   Recalling that the half-rays of the form $(\alpha, \infty)$ generate the
   Borel sets $\Borel$, it follows that $f^{-1}(E) \in \Sigma$ for every Borel
   set $E$ by Lemma \ref{lemma:fn-preimage-generated}. So $f$ is measurable by definition.
\end{proof}

In fact, the condition in Proposition \ref{prop:measurable-fn-r} can be
generalized to any set that generates $\Borel$, such as the open intervals. Now
we provide some examples of measurable functions.

\begin{example}
    A constant function $f:(X,\Sigma_1) \rightarrow (Y, \Sigma_2)$ is
    measurable. 
    To see this for say $f(x) = y$, notice for any $E \in \Sigma_2$ that
    $f^{-1}(E) = X$ if $y \in E$ and $f^{-1}(E) = \emptyset$ if $y \notin E$.
    Clearly $X,\emptyset \in \Sigma_1$, so $f$ is measurable.
\end{example}

\begin{example}
    If $E \in \Sigma$, then define $1_E$ to be the indicator (or characteristic)
    function of $E$; that is, $1_E(x) = 1$ if $x \in E$ and $1_E(x) = 0$
    otherwise. 
    The function $1_E : (X, \Sigma) \rightarrow (\bbR, \Borel)$
    is measurable.
\end{example}

\begin{example}
    Any continuous function $f: (\bbR, \Borel) \rightarrow (\bbR, \Borel)$ is
    measurable: recall that the preimage of any open set under a continuous
    function is open.
\end{example}

\begin{lemma}
    If $f$ and $g$ are measurable real-valued functions and $c \in \bbR$, then
    the following functions are measurable:
    \begin{enumerate}
        \item $fc$
        \item $f + g$
        \item $fg$
        \item $|f|$
    \end{enumerate}
\end{lemma}

For convenience, let $M(X, \Sigma)$ denote the set of measurable functions $f: X
\rightarrow \bar{\bbR}$. The consideration of extended-real-valued functions is
useful when dealing with limits and suprema of sequences, as in
the next lemma.

\begin{lemma}
    Let $(f_n)$ be a sequence in $M(X, \Sigma)$. Then the following functions
    are measurable:

    \begin{enumerate}
        \item $f(x) = \inf f_n(x)$
        \item $F(x) = \sup f_n(x)$
        \item $f^*(x) = \liminf f_n(x)$
        \item $F^*(x) = \limsup f_n(x)$
    \end{enumerate}

\end{lemma}

\begin{proof}

\end{proof}




\newpage

\section{Measures}

\newpage

\section{The Lebesgue Integral}

\end{document}
